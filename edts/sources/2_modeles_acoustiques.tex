\begin{frame}{Traitement acoustique : extraction de paramètres}
	\begin{itemize}
	\item MFCC(Mel Frequency Cepstral Coeficients)
	\item LPCC(Linear Predictive Cepstral Coeficients)
	\item PLP(Perceptuqal Linear Predictive analysis)
	\end{itemize}
\end{frame}

\begin{frame}{Décodage acoustique et apprentissage}

\begin{itemize}
\item Chaînes et modèles de Markov cachés  
\item Critère du maximum de vraisemblance
\item Critère de Viterbi
\end{itemize}

\end{frame}

\begin{frame}{Adaptation}
\begin{itemize}
\item Méthode MLLR
\item Méthode MAP
\end{itemize}

\end{frame}

\begin{frame}{Création du Modèle acoustique en utilisant Sphinxtrain}
Les données d'entrées sont composés, entre autre:
\begin{itemize}
\item d'un ensemble de fichiers acoustiques(corpus).
\item d'un fichier de transcription qui contient l'ensemble de mots prononcés pour chaque enregistrement(fichier acoustique).
\item d'un fichier qui définie la liste des phonèmes utilisées.
\end{itemize}
\end{frame}