\begin{frame}{Le besoin}
	
	\begin{itemize}
		\item Besoin d'une solution open-source permettant de faire de la \emph{Business Intelligence} (BI) Géographique le plus simplement possible.
	~\\
	
\item Objectif: Rendre la \emph{Business Intelligence} Géographique accessible à tous, la rendre naturelle, intuitive.
	\end{itemize}
\end{frame}

\definecolor{light-gray}{gray}{0.65}

\begin{frame}{La \emph{Business Intelligence} (informatique décisionnelle)}
	\begin{itemize}
		\item Données initiales:\\
		Grande quantité de données structurées selon plusieurs dimensions possédant souvent une hiérarchie.\\
		\vspace{0.2cm}
		\pause
		
		{ \itshape \small
			Exemple : Chiffre d'Affaire en fonction de :
			\begin{itemize}
				\item La zone géographique (par pays, région, département);
				\item Le temps (par année, mois, jour);
				\item Le secteur d'activité (par regroupements ou secteur spécifique).
			\end{itemize}
		}
	\end{itemize}
\end{frame}
		
\begin{frame}{La \emph{Business Intelligence} (informatique décisionnelle)}
	
	\begin{itemize}
		\item Traitement :\\
		Sélection et agrégation des données.\\
		\vspace{0.2cm}
		{ \itshape \small
			Exemple : Sélection du CA en France entre 2000 et 2013 pour le secteur d'activité informatique. Agrégation par régions.
		}
		
		\pause
		\vspace{0.25cm}
		\item Visualisation :\\
		Classiquement tableaux et diagrammes. Ajout d'une carte dans la BI Géographique.
	\end{itemize}

\end{frame}

\begin{frame}{Exemple de cas d'utilisation de la BI}
	
	
	\begin{table}
	\small
	\centering
	\begin{tabular}{|r|c|c||c|c|c|}
	\hline
	\multicolumn{3}{|c||}{}                                                & \multicolumn{3}{c|}{Temps} \\ \cline{4-6} 
	\multicolumn{3}{|c||}{}                                                & 2010      & 2011 & 2012    \\ \hline \hline
	\multirow{10}{*}{Géographie} & \multirow{3}{*}{France}   & IDF          & 5         & \cellcolor<2>{green} 4    & \cellcolor<2>{green} 5       \\ \cline{3-6} 
		                        &                           & …            & …         & \cellcolor<2>{green} …    & \cellcolor<2>{green} …       \\ \cline{3-6} 
		                        &                           & Méditerranée & 4         & \cellcolor<2>{green} 3    & \cellcolor<2>{green} 3       \\ \cline{2-6} 
		                        & \multirow{3}{*}{Belgique} & Bruxelles    & 2         & \cellcolor<2>{green} 4    & \cellcolor<2>{green} 3       \\ \cline{3-6} 
		                        &                           & Flandre      & 5         & \cellcolor<2>{green} 1    & \cellcolor<2>{green} 6       \\ \cline{3-6} 
		                        &                           & Wallonie     & 3         & \cellcolor<2>{green} 5    & \cellcolor<2>{green} 5       \\ \cline{2-6} 
		                        & \multirow{3}{*}{Allemagne}& Berlin       & 3         & 2    & 1       \\ \cline{3-6} 
		                        &                           & ...          & 3         & 4    & 5       \\ \cline{3-6} 
		                        &                           & Hamburg      & 6         & 2    & 4       \\ \cline{2-6} 
		                        & …                         & …            & …         & …    & …       \\ \hline
	\end{tabular}
	
	\caption{Exemple de base de données}
	{\small \itshape \only<1>{Données initiales\\~} \only<2>{Sélection des données pour la France et la Belgique\\en 2011 et 2012} }
	
	\end{table}	
\end{frame}

\begin{frame}{Exemple de cas d'utilisation de la BI}

	%\begin{columns}[c]
	
	%\column{6cm}
	
	\begin{table}[h]
	\small 
	\centering
		\begin{tabular}{|c|c||c|}
		\hline
		\multicolumn{2}{|c||}{}                  & 2011-2012 \\ \hline \hline
		\multirow{3}{*}{France}   & IDF          & 9 (4+5)   \\ \cline{2-3} 
	                           & …            & …         \\ \cline{2-3} 
	                           & Méditerranée & 6 (3+3)   \\ \cline{1-3} 
	 \multirow{3}{*}{Belgique} & Bruxelles    & 7 (4+3)   \\ \cline{2-3} 
	                           & Flandre      & 7 (1+6)   \\ \cline{2-3} 
	                           & Wallonie     & 10 (5+5)  \\ \hline
		\end{tabular}
		\caption{Résultats agrégés par région}
	\end{table}
	
	%\column{6cm}
	
	\begin{table}[h]
		\small 
		\centering
		\begin{tabular}{|c||c|c|}
		\hline
        & 2011 				& 2012 				\\ \hline \hline
		France  	& 16 (4+\dots+3) 	& 19 (5+\dots+3)	\\ \hline
	 	Belgique	& 10 (4+1+5)		& 14 (3+6+5)			\\ \hline
		\end{tabular}
		\caption{Résultats agrégés par pays et par année}
	\end{table}
	
	%\end{columns}
	
\end{frame}

\begin{frame}{Exemple de cas d'utilisation de la BI}
	
	\begin{figure}[h]
		\centering
		\begin{tikzpicture}
		\pie[sum=auto, after number=, radius=2]{
			9  / IDF , 10  / … , 6  / Méditerranée , 7  / Bruxelles , 7  / Flandre , 10 / Wallonie
		}
		\end{tikzpicture}
		
		\caption{Visualisation des résultats par diagramme circulaire}
	\end{figure}
\end{frame}

\begin{frame}{Exemple de cas d'utilisation de la BI}
	
	\begin{figure}[h]
		\centering
		\includegraphics[width=7cm]{images/plotBI}
		\caption{Visualisation des résultats par courbes}
	\end{figure}
\end{frame}

\begin{frame}{Exemple de cas d'utilisation de la BI}
	\begin{figure}[h]
		\centering
		\includegraphics[width=8cm]{images/choroplethe}
		\caption{Affichage des données sous forme de carte choroplèthe}
	\end{figure}
	
\end{frame}
