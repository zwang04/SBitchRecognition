\begin{frame}{Objectif}

\begin{itemize}
\item Reconnaissance de parole en utilisant HMM basé sur CMU Sphinx
\end{itemize}

\end{frame}

\begin{frame}{Les Modèle de Markov Caché ou Hidden Markov Model}

\begin{block}{Définition}
Un modèle de Markov caché (MMC) est un modèle
statistique permettant de représenter un processus
de Markov dont l’état est non observable. 
	\end{block}
	\begin{block}{Processus markov(Chaîne de Markov)}
	En mathématiques, un processus de Markov est un processus stochastique possédant la propriété de Markov. Dans un tel processus, la prédiction du futur à partir du présent n'est pas rendue plus précise par des éléments d'information concernant le passé.
	\end{block}
\end{frame}

\begin{frame}{Applications de HMM}
\begin{itemize}
\item Reconnaissance de la parole.
\item Traitement automatique du langage naturel.
\item Reconnaissance de l'écriture manuscrite.
\item Bio-informatique, notamment pour la prédiction de gènes.
\end{itemize}
\end{frame}

\begin{frame}{Un exemple illustratif de HMM}
\begin{figure}
\centering
\includegraphics[width=8cm]{images/hmm.png}
\caption{Un exemple de HMM}
\end{figure}
\end{frame}



\begin{frame}{Un exemple illustratif de HMM}

\begin{figure}
\centering
\includegraphics[width=8cm]{images/hmm1.png}
\caption{Un exemple de HMM}
\end{figure}
\begin{itemize}
\item Les états observés : 1 6 3 5 2 7 3 5 2 4
\item Les états cachés : D6 D8 D8 D6 D4 D8 D6 D4 D8
\end{itemize}

\end{frame}

\begin{frame}{Algorithm de Viterbi}

\begin{itemize}
\item L’algorithme de Viterbi permet de calculer
la valeur la plus probable des états cachés du processus
étant donnée des données observables. 
\end{itemize}

\end{frame}


\begin{frame}{L'Outil CMU Sphinx}
Les CMU Sphinx comprend entre autres les outils suivants:
\begin{itemize}
\item Sphinx 2: est un système de reconnaissance de la parole à grande vitesse. Il est habituellement employé dans des systèmes de dialogue et des système d'étude de prononciation.
\item Sphinx 3: est un système de reconnaissance de la parole légèrement plus lent mais plus précis.
\item Sphinx 4: Une réécriture complète du Sphinx en Java. Il offre à la fois la précision et la rapidité.
\item Sphinxtrain: Une suite d'outils qui permet de créer le modèle acoustique .
\item CMU-Cambridge Language Modeling Toolkit: Une suite d'outils qui permet de créer le modèle de langage.
\item Sphinx Knowledge Base Tool: Un outil qui permet de créer le modèle de mots qui adapte son modèle de langage. 
\end{itemize}

\end{frame}

\begin{frame}{Les Modèle de Markov Caché en speech to text}
Nous considérons un signal acoustique S, le principe de la reconnaissance peut être expliqué comme le calcul de la probabilité P(W|S) qu'une suite de mots (ou phrase) W correspond au signal acoustique S, et de déterminer la suite de mots qui peut maximiser cette probabilité.
En utilisant la formule de Bayes, P(W|S) peut s'écrire:
\begin{figure}
\centering
\includegraphics[width=10cm]{images/hmm_formule.png}
\end{figure}

\end{frame}

\begin{frame}{}

\begin{itemize}
\item P(W) est la probabilité a priori de la suite de mots W
\item P(S|W) est la probabilité de signal acoustique S, étant donné la suite de mots W
\item P(S) est la probabilité du signal acoustique
\item P(S|W) est nommé Modèle Acoustique, et P(W) est nommé Modèle de Langage 
\end{itemize}
\end{frame}

\begin{frame}{Principe de la reconnaissance de la parole}

\begin{figure}
\centering
\includegraphics[width=8cm]{images/principe.png}
\caption{Principe de la reconnaissance de la parole}
\end{figure}
\end{frame}
