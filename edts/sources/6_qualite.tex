\begin{frame}{La démarche qualité}
\begin{figure}
	\centering\includegraphics[width=8cm]{./images/hierarchie.pdf}
\caption{Hiérarchie partielle des documents applicables}
\end{figure}
	\end{frame}
	
	\begin{frame}{Système de Management de la Qualité}
	\begin{itemize}
		\item Mise en place du SMQ ;
		\item Suivi de la qualité ;
		\item Amélioration continue du SMQ.
	\end{itemize}
	\end{frame}
	
	\begin{frame}{Mise en place du SMQ}
	\begin{itemize}
	 \item Réalisation d'une Réunion Formelle de Démarrage (RFD) ;
	 \pause
	 \item Rédaction du Plan Qualité (PQ);
	 \item Rédaction du Plan de Gestion des Configurations (PGC).
	\end{itemize}
	\end{frame}
	
	\begin{frame}{Gestion des indicateurs}
	\begin{itemize}
		\pause
		\item Objectifs
		\begin{itemize}
		 \item assurer l'avancement du projet ;
		 \item assurer le respect des délais ;
		 \item assurer la conformité des produits ;
		 \item assurer l'efficience du traitement des fiches de faits techniques ;
		 \item assurer une bonne communication entre les différentes parties prenantes du projet.
		\end{itemize}
		\pause
		\item Tableau de bord
		\begin{itemize}
		 \item visualisation des indicateurs ;
		 \item diffusion des informations relevées.
		\end{itemize}
	\end{itemize}
	\end{frame}
	
	\begin{frame}{Exemple d'indicateur}
\begin{figure}
\centering
\includegraphics[width=5.3cm]{./images/volumehoraire1.pdf}
\includegraphics[width=5.3cm]{./images/volumehoraire2.pdf}
\caption{Exemple d'indicateur, volume horaire des membres de l'équipe}
\end{figure}

	\end{frame}	
	
	\begin{frame}{Gestion des risques}
	\'Etapes de la gestion :
	\begin{itemize}

		 \item Identification des risques ;
		 \item Suivi des risques ;
		 \item Réduction des risques.
		\end{itemize}
	\pause
	Exemple de risque :\\
		\begin{center}
	\begin{tabular}{|c|c|}
	\hline
	Numéro du risque & 004 \\ \hline
	Nom du risque & Mauvaise planification \\ \hline
	Gravité & Importante \\ \hline
	Probabilité & Très probable \\ \hline
	Criticité & Très haute \\ \hline
	\end{tabular}
	\end{center}
	
	\end{frame}	
	
	\begin{frame}{Suivi de la qualité}
\begin{figure}
	\centering
	\includegraphics[width=5cm]{./images/suivi_qualite.pdf}
	\caption{Représentation du suivi de la qualité}
	\end{figure}
	
%	\begin{itemize}
%	%une image serait mieux
%	 \item Gestion des indicateurs ;
%	 \item Mise à jour des tableaux de bord ;
%	 \item Suivi des risques ;
%	 \item Gestion des faits techniques.
%	\end{itemize}
	\end{frame}
	
	\begin{frame}{Gestion des faits techniques}
\begin{figure}
	\centering
	\includegraphics[width=10cm]{./images/CycleFT_EpicUnity.pdf}
	\caption{Cycle correctif}
	\end{figure}
	\end{frame}
	
	\begin{frame}{Surveillance de la qualité}
	\begin{itemize}
	 \item Audits internes ;
	 \item Audits de code ;
	 \item Audits externes .
	\end{itemize}
	\end{frame}
	
	\begin{frame}{Amélioration continue du SMQ}
	\begin{itemize}
	 \item Correction de la politique qualité mise en place;
	 \item Maintenance et mise à jour des documents relatifs à la qualité.
	\end{itemize}
	\end{frame}
