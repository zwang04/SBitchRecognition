\begin{frame}{Découverte des technologies}

\begin{itemize}
 \item Formation intensive (300 heures-homme);
 \item Acquisition de livres;
 \item Approche par l'exemple;
 \item Appui sur un travail existant : SEMUNIT de Madame Yolaine Bourda.
\end{itemize}

\end{frame}

\begin{frame}{Résultat produit (1/2)}

\begin{itemize}
 \item Un schéma OWL structuré selon le SupLOMFR;
 \item Le vocabulaire SKOS est séparé;
 \item Respect des cardinalités;
 \item Évolutions sur le prochain lot (au niveau de l'inférence).

\end{itemize}
\end{frame}

\begin{frame}{Résultat produit (2/2)}
\begin{itemize}
 \item Des fiches en XML-RDF :
 \begin{itemize}
    \item Issues d'un transformateur opérationnel;
    \item Contenant les triplets RDF d'une fiche;
  \end{itemize}
 \item Vocabulaire externalisé.
\end{itemize}
\end{frame}

\begin{frame}{Garantie de fonctionnement}
\begin{itemize}
 \item Syntaxe, contenu et cohérence;
 \item Compatibilité entre le schéma et la structure des fiches XML;
 \item Une campagne de tests complète.
\end{itemize}
\end{frame}


\begin{frame}{Stratégie de tests}
	\begin{figure}
	  \centering
	  \includegraphics[width=8cm]{images/tests.pdf}
	  \caption{Schéma de principe des tests}
	\end{figure}
\end{frame}

\begin{frame}{Choix des outils de test}

\begin{tiny}
\begin{table}[h]
\begin{tabular}{|c|c|c|c|} \cline{1-4}
Framework & Désordre des triplets & Contenu des balises & \begin{tabular}[c]{@{}c@{}}Bonne formation \\ des triplets\end{tabular}\\ \cline{1-4}
Protégé &  &  & \\ \cline{1-4}
RDFUnit &  &  & \\ \cline{1-4}
XSPEC &  & \checkmark{} & \checkmark{} \\ \cline{1-4}
XSLTUnit &  &  & \checkmark{} \\ \cline{1-4}
Unit testing XSLT &  & \checkmark{} & \checkmark{}\\ \cline{1-4}
\color{red}{Développement par l'équipe} & \color{red}{\checkmark{}} & \color{red}{\checkmark{}} & \color{red}{\checkmark{}} \\ \cline{1-4}
\end{tabular}
\end{table}
\end{tiny}

\begin{itemize}
 \item Des solutions développées par l'équipe;
 \item XML-RDF : xmllint + Python (basé sur rdflib);
 \item OWL : W3C, vérifications manuelles et un raisonneur.
\end{itemize}

\end{frame}


\begin{frame}{Volumétrie}

\begin{figure}
    \centering
    \includegraphics[width=7cm]{images/volumetrie.pdf}
    \caption{Transformation de fiches distinctes}
\begin{tiny}
\begin{table}[h]
\begin{tabular}{|c|c|c|c|c|c|c|c|c|}
\cline{1-9}
nombre de fiches & 1 & 2 & 10 & 50 & 100 & 500 & 1000 & 5000\\ \cline{1-9}
temps de conversion (s) & 1,14 & 2,3 & 11,68 & 58,04 & 116,21 & 579,63 & 1158,89 & 5794,56 \\ \cline{1-9}
\end{tabular}
\end{table}
\end{tiny}
\tablenameUn{}
  \end{figure}

\end{frame}

\begin{frame}{Démonstration}

Démonstration de requêtes SPARQL.
\centering
\includemovie[controls,poster]{6cm}{6cm}{video/demo.mp4}

\end{frame}
