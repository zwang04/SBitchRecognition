\documentclass[compress,xcolor=table]{beamer}

\usepackage[french]{babel}
\selectlanguage{french}
\usepackage[utf8]{inputenc}
\usepackage[T1]{fontenc}
\usepackage{tikz}
\usepackage{wrapfig}
\usepackage{multirow}
\usepackage{pgf-pie}
\usepackage{pgfplots}
\usepackage{pdfpages}
\usepackage{vocabulaireEpicUnityPresentation}
\usepackage{commun/vocabulaireCommun}
\usepackage{hyperref}
\usepackage{movie15}

\usetheme{eastpic}

% code pour pouvoir mettre des cellcolor qui dépendent de la frame...
\makeatletter
\def\rowcolor{\noalign{\ifnum0=`}\fi\bmr@rowcolor}
\newcommand<>{\bmr@rowcolor}{%
    \alt#1%
	{\global\let\CT@do@color\CT@@do@color\@ifnextchar[\CT@rowa\CT@rowb}% 
	{\ifnum0=`{\fi}\@gooble@rowcolor}% 
}

\newcommand{\@gooble@rowcolor}[2][]{\@gooble@rowcolor@}
\newcommand{\@gooble@rowcolor@}[1][]{\@gooble@rowcolor@@}
\newcommand{\@gooble@rowcolor@@}[1][]{\ignorespaces}
\makeatother



\makeatletter
\def\cellcolor{{\ifnum0=`}\fi\bmr@cellcolor}
\newcommand<>{\bmr@cellcolor}{%
    \alt#1%
	{\global\let\CT@do@color\CT@@do@color\@ifnextchar[\CT@rowa\CT@rowb}%
	 {\ifnum0=`{\fi}\@gooble@cellcolor}%
}

\newcommand{\@gooble@cellcolor}[2][]{\@gooble@cellcolor@}
\newcommand{\@gooble@cellcolor@}[1][]{\@gooble@cellcolor@@}
\newcommand{\@gooble@cellcolor@@}[1][]{\ignorespaces}

\newcommand{\tablenameUn}{Tableau 1 : Temps de conversion en fonction du nombre de fiches}



\def\sectionintoc{}
\def\beamer@sectionintoc#1#2#3#4#5{%
\ifnum\c@tocdepth>0%
\ifnum#4=\beamer@showpartnumber%
{
  \beamer@saveanother%
  \gdef\beamer@todo{}%
  \beamer@slideinframe=#1\relax%
  \expandafter\only\beamer@tocsections{\gdef\beamer@todo{%
      \beamer@tempcount=#5\relax%
      \advance\beamer@tempcount by\beamer@sectionadjust%
      \edef\inserttocsectionnumber{\the\beamer@tempcount}%
      \def\inserttocsection{\hyperlink{Navigation#3}{#2}}%
      \beamer@tocifnothide{\ifnum\c@section=#1\beamer@toc@cs\else\beamer@toc@os\fi}%
      {
        \ifbeamer@pausesections\pause\fi%
        \ifx\beamer@toc@ooss\beamer@hidetext
          \vskip1em
        \else
          \vfill
        \fi
        {%
          \hbox{\vbox{%
              \def\beamer@breakhere{\\}%
              \beamer@tocact{\ifnum\c@section=#1\beamer@toc@cs\else\beamer@toc@os\fi}    {section in toc}}}%
         \par%
        }%
      }%
    }
  }%
  \beamer@restoreanother%
  }
  \beamer@todo%
  \fi\fi%
}



\makeatother
% end code pour pouvoir mettre des cellcolor qui dépendent de la frame


\title{Présentation de projet EDTS - Reconnaissance de parole en utilisant HMM basé sur CMU Sphinx}
\date{\today}
\author{Zhaolun Wang et Zenan Xu}
\institute{\insa}

\setbeameroption{show notes}

\begin{document}


\begin{frame}[plain]
	\titlepage
\end{frame}

\begin{frame}{Sommaire}
	\tableofcontents[hideallsubsections]
\end{frame}

\section[Introduction]{Introduction}
\subsection{}

\begin{frame}{Le client, Fondation UNIT}

\begin{figure}
\centering
\includegraphics[width=5cm]{images/logo.png}
\caption{Logo de la fondation UNIT}
\end{figure}

\begin{itemize}
  \pause
  \item UNIT: l'Université Numérique Ingénierie et Technologie ;
  \pause
  \item Fondation partenariale d'environ 70 Universités, Grandes Ecoles d'Ingénieurs et Entreprises ;
  \pause
  \item Une des sept universités numériques thématiques (UNT) nationales.

\end{itemize}
	
\end{frame}

\begin{frame}{L'équipe \textsc{EpicUnity}}
	
	\begin{figure}
	\centering
	\includegraphics[width=10cm]{images/organigramme.png}
	\caption{Organigramme de l'équipe}
	\end{figure}
	
\end{frame}

\section[Principe de la reconnaissance de la parole]{Modèles acoustiques}
\subsection{}
\begin{frame}{Les Modèle de Markov Caché(MMC) ou Hidden Markov Model(HMM)}

\begin{block}{Définition}
Un modèle de Markov caché (MMC) est un modèle
statistique permettant de représenter un processus
de Markov dont l’état est non observable. 
	\end{block}
	\begin{block}{Processus markov(Chaîne de Markov)}
	En mathématiques, un processus de Markov est un processus stochastique possédant la propriété de Markov. Dans un tel processus, la prédiction du futur à partir du présent n'est pas rendue plus précise par des éléments d'information concernant le passé.
	\end{block}
\end{frame}


\begin{frame}{Un exemple illustratif de HMM}
\begin{figure}
\centering
\includegraphics[width=8cm]{images/hmm.png}
\caption{Un exemple de HMM}
\end{figure}
\end{frame}



\begin{frame}{Un exemple illustratif de HMM}

\begin{figure}
\centering
\includegraphics[width=8cm]{images/hmm1.png}
\caption{Un exemple de HMM}
\end{figure}
\begin{itemize}
\item Les états observés : 1 6 3 5 2 7 3 5 2 4
\item Les états cachés : D6 D8 D8 D6 D4 D8 D6 D4 D8
\end{itemize}

\end{frame}

\begin{frame}{Algorithm de Viterbi}

\begin{itemize}
\item L’algorithme de Viterbi permet de calculer
la valeur la plus probable des états cachés du processus
étant donneê des données observables. 
\end{itemize}

\end{frame}

\begin{frame}{Modèle acoustique}

	
\end{frame}

%\section[Principe de la reconnaissance de la parole]{Modèles de langues}
\subsection{}
\begin{frame}{Modèle des N-Grammes}
\begin{figure}
\centering
\includegraphics[width=8cm]{images/modele_langage.png}
\end{figure}
\end{frame}

\begin{frame}{Création de Modèle de langage en utilisant CMUCLMTK}
La création d'un Modèle de langage statistique peut se résumer en trois étapes:
\begin{itemize}
\item Collecter des textes
\item Transformer les textes en corpus
\item Transformer le corpus en une distribution de probabilités
\end{itemize}
\end{frame}



\section[Principe de la reconnaissance de la parole]{Modèles de mots}
\subsection{}
\input{sources/4_modeles_de_mots}

\section[Réalisation]{Réalisation du projet}
\subsection{}
%\begin{frame}{Découverte des technologies}

\begin{itemize}
 \item Formation intensive (300 heures-homme);
 \item Acquisition de livres;
 \item Approche par l'exemple;
 \item Appui sur un travail existant : SEMUNIT de Madame Yolaine Bourda.
\end{itemize}

\end{frame}

\begin{frame}{Résultat produit (1/2)}

\begin{itemize}
 \item Un schéma OWL structuré selon le SupLOMFR;
 \item Le vocabulaire SKOS est séparé;
 \item Respect des cardinalités;
 \item Évolutions sur le prochain lot (au niveau de l'inférence).

\end{itemize}
\end{frame}

\begin{frame}{Résultat produit (2/2)}
\begin{itemize}
 \item Des fiches en XML-RDF :
 \begin{itemize}
    \item Issues d'un transformateur opérationnel;
    \item Contenant les triplets RDF d'une fiche;
  \end{itemize}
 \item Vocabulaire externalisé.
\end{itemize}
\end{frame}

\begin{frame}{Garantie de fonctionnement}
\begin{itemize}
 \item Syntaxe, contenu et cohérence;
 \item Compatibilité entre le schéma et la structure des fiches XML;
 \item Une campagne de tests complète.
\end{itemize}
\end{frame}


\begin{frame}{Stratégie de tests}
	\begin{figure}
	  \centering
	  \includegraphics[width=8cm]{images/tests.pdf}
	  \caption{Schéma de principe des tests}
	\end{figure}
\end{frame}

\begin{frame}{Choix des outils de test}

\begin{tiny}
\begin{table}[h]
\begin{tabular}{|c|c|c|c|} \cline{1-4}
Framework & Désordre des triplets & Contenu des balises & \begin{tabular}[c]{@{}c@{}}Bonne formation \\ des triplets\end{tabular}\\ \cline{1-4}
Protégé &  &  & \\ \cline{1-4}
RDFUnit &  &  & \\ \cline{1-4}
XSPEC &  & \checkmark{} & \checkmark{} \\ \cline{1-4}
XSLTUnit &  &  & \checkmark{} \\ \cline{1-4}
Unit testing XSLT &  & \checkmark{} & \checkmark{}\\ \cline{1-4}
\color{red}{Développement par l'équipe} & \color{red}{\checkmark{}} & \color{red}{\checkmark{}} & \color{red}{\checkmark{}} \\ \cline{1-4}
\end{tabular}
\end{table}
\end{tiny}

\begin{itemize}
 \item Des solutions développées par l'équipe;
 \item XML-RDF : xmllint + Python (basé sur rdflib);
 \item OWL : W3C, vérifications manuelles et un raisonneur.
\end{itemize}

\end{frame}


\begin{frame}{Volumétrie}

\begin{figure}
    \centering
    \includegraphics[width=7cm]{images/volumetrie.pdf}
    \caption{Transformation de fiches distinctes}
\begin{tiny}
\begin{table}[h]
\begin{tabular}{|c|c|c|c|c|c|c|c|c|}
\cline{1-9}
nombre de fiches & 1 & 2 & 10 & 50 & 100 & 500 & 1000 & 5000\\ \cline{1-9}
temps de conversion (s) & 1,14 & 2,3 & 11,68 & 58,04 & 116,21 & 579,63 & 1158,89 & 5794,56 \\ \cline{1-9}
\end{tabular}
\end{table}
\end{tiny}
\tablenameUn{}
  \end{figure}

\end{frame}

\begin{frame}{Démonstration}

Démonstration de requêtes SPARQL.
\centering
\includemovie[controls,poster]{6cm}{6cm}{video/demo.mp4}

\end{frame}


\section[Amélioration]{Amélioration}
\subsection{}

\section[Conclusion]{Conclusion}
\subsection{}

%\begin{frame}{Conclusion}
	\begin{itemize}
	\item Démarche enrichissante :
		\begin{itemize}
			\item Développement en méthodes de type agile;
			\item Démarche Qualité.
		\end{itemize}
		\vspace{0.2cm}
	\item Sujet passionnant :
		\begin{itemize}
			\item Découverte web sémantique ;
			\item Formation et perfectionnement dans différents domaines;
			\item Démarche de R\&D novatrice dans le domaine.
		\end{itemize}		
	\item Vers l'inférence et la gestion des personnes.
	\end{itemize}
\end{frame}

\end{document}






